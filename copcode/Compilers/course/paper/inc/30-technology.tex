\section{Технологическая часть}

\subsection{Обоснование средств программной реализации}
В качестве языка программирования для написания компилятора был выбран Go\cite{golang}, ввиду следующих причин.
\begin{itemize}
    \item Имеется личный опыт работы с данным языком на других курсовых проектах.
    \item ANTLR4\cite{golang:antlr4} и LLVM\cite{golang:llvm} поддерживают библиотеки на Go. По моей субъективной оценке в данном языке их использование будет наиболее удобным.
\end{itemize}

\subsection{Описание программы}
Обход синтаксического дерева в бибиотеке antlr4-go можно осуществить с помощью паттерна Walker или Visitor. Первый состоит в описании методов enter и exit для каждого типа узла дерева. Второй - в описании методов visit для каждого типа узла. Был выбран обход с помощью Visitor, так как он позволяет контролировать порядок вызова обработки дочерних узлов, а также поддерживает возврат значения.

LLVM IR формируется при помощи структуры Module, предоставляющей интерфейс для построения промежуточного представления. Эта и другие переменные контекста (текущий блок, текущая функция, области видимости переменных) хранятся в Visitor и используются по мере обхода дерева.

В результате обхода дерева получается заполненная структура LLVM модуля, которая записывается в текстовом формате в .ll файл. После чего компилируется с помощью утилит \textbf{clang} и \textbf{lld-link}.

\subsection{Тестирование программы}
Для тестирования программы были написаны программы на языке Си, удовлетворяющие ранее сформированной грамматике. Тестирование производится в три этапа.
\begin{enumerate}
    \item Тестовая программа компилируется при помощи написанного модуля.
    \item Тестовая программа компилируется при помощи gcc.
    \item Полученные исполняемые файлы запускаются. Сравниваются коды возврата данных программ.
\end{enumerate}

Так как грамматика не предусматривает возможность использования функций ввода/вывода из стандартных библиотек Си, единственным путём проверки результатов вычислений остаётся код возврата программмы. Стоит отметить, что такое использование кода возврата не соответствует его основной функции - передачи кода ошибки, с которым завершилась программа. Однако, в условиях данного проекта такое использование возвращаемого значения в целях демонстрации работы программы было сочтено приемлимым.

\subsection{Пример работы программы}
Для примера работы программы используется один из тестовых примеров - вычисление n-го числа Фибоначчи с использованием статического массива. В следующих листингах \ref{example:fib:c} \ref{example:fib:ll}.

\lstinputlisting[caption = Пример программы]{code/fibonacci.c}\label{example:fib:c}
\lstinputlisting[caption = Пример LLVM IR]{code/fibonacci.ll}\label{example:fib:ll}

\pagebreak
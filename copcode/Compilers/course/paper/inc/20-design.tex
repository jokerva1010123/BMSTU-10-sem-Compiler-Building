\section{Конструкторская часть}

\subsection{IDEF0}
Концептуальная модель программы представлена в нотации IDEF0 на рисунке \ref{pic:idef0-1}

\myImage
{schemes/idef0.pdf}
{Концептуальная модуль системы в нотации IDEF0.}
{pic:idef0-1}

\subsection{Грамматика языка tinyc}
Грамматика языка tinyc \ref{lst:grammar} является крайне упрощённым вариантом грамматики языка Си.

\lstinputlisting[caption = Грамматика языка tinyc]{code/tinyc.g4}\label{lst:grammar}

Так как эта грамматика не удовлетворяла требованиям курсовой работы, она была дополнена элементами грамматики Си. Были добавлены следующие элементы синтаксиса:
\begin{itemize}
    \item недостающие арифметические операции с учётом приоритетов действий;
    \item объявления и инициализация переменных;
    \item поддержка многомерных статических массивов;
    \item комментарии.
\end{itemize}
Полная дополненная грамматика приведена в Приложении А.

\subsection{Обход синтаксического дерева}
Исходная программа преобразовывается в синтаксическое дерево при помощи кода, сгенерированного ANTLR4 для описанной грамматики. Обход всех узлов данного представления позволяет сгенерировать LLVM IR.

Рассмотрим синтаксическое дерево на примере изображения из Приложения Б. Данная визуализация получена использованием утилиты \textbf{antlr4-parse} для следующей программы, вычисляющей остаток от деления числа на 7.

\lstinputlisting[caption = Пример программы (остаток от деления на 7)]{code/div7.c}\label{lst:fib}

\subsection{Генерация LLVM IR}
Сгенерировать промежуточное представление LLVM можно путём обхода всех узлов синтаксического дерева. Каждый узел может создавать новые инструкции, блоки, функции и т.п. в зависимости от его типа и дочерних узлов. Рассмотрим пример обработки узла \textbf{iterationStatement}, грамматика которого показана в выражении \ref{iterationStatement}.

\begin{equation}
    \label{iterationStatement}
    \text{iterationStatement: While '(' expression ')' statement;}
\end{equation}

Пример LLVM IR представления, сгенерированного данным алгоритмом по вышеупомянутой программе приведён в листинге \ref{lst:div7.ll}.

\lstinputlisting[caption = Пример LLVM IR (остаток от деления на 7)]{code/div7.ll}\label{lst:div7.ll}

На рисунке \ref{pic:while} приведена схема алгоритма генерации кода для цикла while.

\myImage
{schemes/while.pdf}
{Схема алгоритма обработки узла iterationStatement.}
{pic:while}

\subsection*{Вывод}
В данном разделе была предоставлена концептуальная модель метода компиляции в нотации диаграммы IDEF0. Приведена грамматика tinyc, описаны предпринятые расширения. Описаны работы frontend часть разрабатываемого компилятора.

\pagebreak

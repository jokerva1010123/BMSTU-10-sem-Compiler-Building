\documentclass{bmstu-gost-7-32}

\begin{document}

\makereporttitle
	{Информатика, искусственный интеллект и системы управления} % Название факультета
	{Программное обеспечение ЭВМ и информационные технологии} % Название кафедры
	{лабораторной работе №~4} % Название работы (в дат. падеже)
	{Конструирование компиляторов} % Название курса (необязательный аргумент)
	{Синтаксический анализатор операторного предшествования} % Тема работы
	{3} % Номер варианта (необязательный аргумент)
	{ИУ7-23М} % Номер группы
	{Волкова~А.~А.} % ФИО студента
	{Ступников~А.~А.} % ФИО преподавателя

\section*{Описание задания}

\textbf{Цель работы:} приобретение практических навыков реализации таблично управляемых синтаксических анализаторов на примере анализатора операторного предшествования.

\textbf{Задачи работы:}
\begin{enumerate}
	\item Ознакомиться с основными понятиями и определениями, лежащими в основе синтаксического анализа операторного предшествования.
	\item Изучить алгоритм синтаксического анализа операторного предшествования.
	\item Разработать, тестировать и отладить программу синтаксического анализа в соответствии с предложенным вариантом грамматики.
	\item Включить в программу синтаксического анализ семантические действия для реализации синтаксически управляемого перевода инфиксного выражения в обратную польскую нотацию.
\end{enumerate}

\textbf{Вариант 3. Грамматика G3}

Рассматривается грамматика выражений отношения с правилами

\begin{verbatim}
	<выражение> ->
	    <арифметическое выражение> <знак операции отношения> <арифметическое выражение>

	<арифметическое выражение> ->
	    <терм> |
	    <знак операции типа сложения> <терм> |
	    <арифметическое выражение> <знак операции типа сложения> <терм>

	<терм> ->
	    <множитель> |
	    <терм> <знак операции типа умножения> <множитель>

	<множитель> ->
	    <первичное выражение> |
	    <множитель> ^ <первичное выражение>

	<первичное выражение> ->
	    <число> |
	    <идентификатор> |
	    ( <арифметическое выражение> )

	<знак операции типа сложения> ->
	    + | -

	<знак операции типа умножения> ->
	    * | / | %

	<знак операции отношения> ->
	    < | <= | = | >= | > | <>
\end{verbatim}

\textbf{Замечания.}

\begin{enumerate}
	\item Нетерминалы <идентификатор> и <число> — это лексические единицы (лексемы), которые оставлены неопределенными, а при выполнении лабораторной работы можно либо рассматривать их как терминальные символы, либо определить их по своему усмотрению и добавить эти определения.
	\item Терминалы ( ) — это разделители и символы пунктуации.
	\item Терминалы + - * / \% < <= = >= > <> — это знаки операций.
	\item Нетерминал <выражение> — это начальный символ грамматики.
\end{enumerate}

\section*{Текст программы}

С полным текстом программы можно ознакомиться по адресу: \url{https://github.com/Volkovaan/CD/tree/lab04/lab04/src}.

\section*{Тестирование и результаты}

Выражение:
\begin{verbatim}
	a + c * ( b - c ) / 1
\end{verbatim}

Результат:
\begin{verbatim}
	a c + b c - * 1 /
\end{verbatim}

\section*{Выводы}

В данной лабораторной работе были изучены основные понятия и определения, лежащие в основе синтаксического анализа операторного предшествия, изучен алгоритм синтаксического анализа операторного предшествия, разработана, протестирована и отлажена программа синтаксического анализа в соответствии с предложенным вариантом грамматики.
В программу синтаксического анализа были включены семантические действия для реализации синтаксически управляемого перевода инфиксного выражения в обратную польскую нотацию.

\end{document}

\documentclass{bmstu-gost-7-32}

\begin{document}

\makereporttitle
	{Информатика, искусственный интеллект и системы управления} % Название факультета
	{Программное обеспечение ЭВМ и информационные технологии} % Название кафедры
	{лабораторной работе №~1} % Название работы (в дат. падеже)
	{Конструирование компиляторов} % Название курса (необязательный аргумент)
	{Распознавание цепочек регулярного языка} % Тема работы
	{3} % Номер варианта (необязательный аргумент)
	{ИУ7-23М} % Номер группы
	{Волкова~А.~А.} % ФИО студента
	{Ступников~А.~А.} % ФИО преподавателя

\section*{Описание задания}

\textbf{Цель работы:} приобретение практических навыков реализации важнейших элементов лексических анализаторов на примере распознавания цепочек регулярного языка.

\textbf{Задачи работы:}
\begin{enumerate}
	\item Ознакомиться с основными понятиями и определениями, лежащими в основе построения лексических анализаторов.
	\item Прояснить связь между регулярным множеством, регулярным выражением, праволинейным языком, конечно-автоматным языком и недетерминированным конечно-автоматным языком.
	\item Разработать, тестировать и отладить программу распознавания цепочек регулярного или праволинейного языка в соответствии с предложенным вариантом грамматики.
\end{enumerate}

\textbf{Вариант 3}

Напишите программу, которая в качестве входа принимает произвольное регулярное выражение, и выполняет следующие преобразования:

\begin{enumerate}
	\item По регулярному выражению строит НКА.
	\item По НКА строит эквивалентный ему ДКА.
	\item По ДКА строит эквивалентный ему КА, имеющий наименьшее возможное количество состояний.

	\textit{Указание.} Воспользоваться алгоритмом, приведённым по адресу \href{http://neerc.ifmo.ru/wiki/index.php?title=\%D0\%9C\%D0\%B8\%D0\%BD\%D0\%B8\%D0\%BC\%D0\%B8\%D0\%B7\%D0\%B0\%D1\%86\%D0\%B8\%D1\%8F_\%D0\%94\%D0\%9A\%D0\%90,_\%D0\%B0\%D0\%BB\%D0\%B3\%D0\%BE\%D1\%80\%D0\%B8\%D1\%82\%D0\%BC_\%D0\%A5\%D0\%BE\%D0\%BF\%D0\%BA\%D1\%80\%D0\%BE\%D1\%84\%D1\%82\%D0\%B0_(\%D1\%81\%D0\%BB\%D0\%BE\%D0\%B6\%D0\%BD\%D0\%BE\%D1\%81\%D1\%82\%D1\%8C_O(n_log_n))}{http://neerc.ifmo.ru/wiki/index.php?title=Минимизация\_ДКА,\\\_алгоритм\_Хопкрофта\_(сложность\_O(n\_log\_n))}

	\item Моделирует минимальный КА для входной цепочки из терминалов исходной грамматики.
\end{enumerate}

\section*{Текст программы}

С полным текстом программы можно ознакомиться по адресу: \url{https://github.com/Volkovaan/CD/tree/lab01/lab01/src}.

\section*{Тестирование и результаты}

Для тестирования использовались примеры 3.21, 3.24 из [3] а также был написан ряд тестов, проверяющих корректное создание НКА по регулярным выражениям.

Регулярные выражения и пары (тест, результат):
\begin{itemize}
	\item $(a|b)^*abb$ — $(a, 0)$, $(b, 0)$, $(ab, 0)$, $(aab, 0)$, $(abb, 1)$, $(aabb, 1)$
	\item $a^*|b$ — $(\varepsilon, 1)$, $(a, 1)$, $(aaaa, 1)$, $(b, 1)$, $(bbb, 0)$, $(ab, 0)$
	\item $((((ab))))$ — $(\varepsilon, 0)$, $(a, 0)$, $(b, 0)$, $(ab, 1)$, $(abb, 0)$
	\item $((((a))^*)^*)$ — $(\varepsilon, 1)$, $(a, 1)$, $(aaaa, 1)$, $(ab, 0)$, $(b, 0)$
	\item $((a|bb)^*(a|bb)b*)|b$ — $(\varepsilon, 0)$, $(a, 1)$, $(b, 1)$, $(aa, 1)$, $(bb, 1)$, $(ab, 1)$, $(ba, 0)$, $(abbabab, 0)$, $(aaa, 1)$, $(abbabbb, 1)$
\end{itemize}

\section*{Выводы}

В данной лабораторной работе были приобретены практические навыки реализации важнейших элементов лексических анализаторов на примере распознавания цепочек регулярного языка.
Были проработаны основные понятия и определения, лежащие в основе построения лексических анализаторов и прояснены связи между регулярным множеством, регулярным выражением, праволинейным языком, конечно-автоматным языком и недетерминированным конечно-автоматным языком. Результатом работы является разработанная программа распознавания цепочек регулярного языка в соответствии с предложенным вариантом грамматики.

\section*{Список использованной литературы}

\begin{enumerate}
	\item Белоусов А. И., Ткачёв С. Б. Дискретная математика: Учеб. Для вузов / Под ред. В. С. Зарубина, А.П. Крищенко. – М.: Изд-во МГТУ им. Н. Э. Баумана, 2021.
	\item Минимизация ДКА, алгоритм Хопкрофта (сложность O(n log n)). Университет ИТМО, URL: \href{http://neerc.ifmo.ru/wiki/index.php?title=\%D0\%9C\%D0\%B8\%D0\%BD\%D0\%B8\%D0\%BC\%D0\%B8\%D0\%B7\%D0\%B0\%D1\%86\%D0\%B8\%D1\%8F_\%D0\%94\%D0\%9A\%D0\%90,_\%D0\%B0\%D0\%BB\%D0\%B3\%D0\%BE\%D1\%80\%D0\%B8\%D1\%82\%D0\%BC_\%D0\%A5\%D0\%BE\%D0\%BF\%D0\%BA\%D1\%80\%D0\%BE\%D1\%84\%D1\%82\%D0\%B0_(\%D1\%81\%D0\%BB\%D0\%BE\%D0\%B6\%D0\%BD\%D0\%BE\%D1\%81\%D1\%82\%D1\%8C_O(n_log_n))}{http://neerc.ifmo.ru/wiki/index.php?title=Минимизация\_ДКА,\\\_алгоритм\_Хопкрофта\_(сложность\_O(n\_log\_n))}
	\item Ахо А. В, Лам М. С., Сети Р., Ульман Дж. Д. Компиляторы: принципы, технологии и инструменты. – М.: Вильямс, 2008.
\end{enumerate}

\end{document}

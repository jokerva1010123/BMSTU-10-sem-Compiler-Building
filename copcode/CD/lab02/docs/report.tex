\documentclass{bmstu-gost-7-32}

\begin{document}

\makereporttitle
	{Информатика, искусственный интеллект и системы управления} % Название факультета
	{Программное обеспечение ЭВМ и информационные технологии} % Название кафедры
	{лабораторной работе №~2} % Название работы (в дат. падеже)
	{Конструирование компиляторов} % Название курса (необязательный аргумент)
	{Преобразование грамматик} % Тема работы
	{3} % Номер варианта (необязательный аргумент)
	{ИУ7-23М} % Номер группы
	{Волкова~А.~А.} % ФИО студента
	{Ступников~А.~А.} % ФИО преподавателя

\section*{Описание задания}

\textbf{Цель работы:} приобретение практических навыков реализации наиболее важных (но не всех) видов преобразования грамматик, чтобы удовлетворить требованиям алгоритмов синтаксического разбора.

\textbf{Задачи работы:}
\begin{enumerate}
	\item Принять к сведению соглашения об обозначениях, принятые в литературе по теории формальных языков и грамматик и кратко описанные в приложении.
	\item Познакомиться с основными понятиями и определениями теории формальных языков и грамматик.
	\item Детально разобраться в алгоритме устранения левой рекурсии.
	\item Разработать, тестировать и отладить программу устранения левой рекурсии.
	\item Разработать, тестировать и отладить программу преобразования грамматики в соответствии с предложенным вариантом.
\end{enumerate}

\textbf{Общий вариант для всех. Устранение левой рекурсии}

\textbf{Определение.}
Нетерминал $A$ КС-грамматики $G = (N, \Sigma, P, S)$ называется рекурсивным, если $A \Rightarrow^+ \alpha A \beta$ для некоторых $\alpha$ и $\beta$.
Если $\alpha = \varepsilon$, то $A$ называется леворекурсивным.
Аналогично, если $\beta = \varepsilon$, то $A$ называется праворекурсивным.
Грамматика, имеющая хотя бы один леворекурсивный нетерминал, называется леворекурсивной.
Аналогично определяется праворекурсивная грамматика.
Грамматика, в которой все нетерминалы, кроме, быть может, начального символа, рекурсивные, называется рекурсивной.

Некоторые из алгоритмов разбора не могут работать с леворекурсивными грамматиками.
Можно показать, что каждый КС-язык определяется хотя бы одной не леворекурсивной грамматикой.

Постройте программу, которая в качестве входа принимает приведённую КС-грамматику $G = (N, \Sigma, P, S)$ и преобразует её в эквивалентную КС-грамматику $G'$ без левой рекурсии.

\textit{Указания.}
\begin{enumerate}
	\item Проработать самостоятельно п. 4.3.3. и п. 4.3.4. [2].
	\item Воспользоваться алгоритмом 2.13.
	При тестировании воспользоваться примером 2.27. [1].
	\item Воспользоваться алгоритмами 4.8 и 4.10.
	При тестировании воспользоваться примерами 4.7., 4.9. и 4.11. [2].
	\item Устранять надо не только непосредственную (immediate), но и косвенную (indirect) рекурсию.
	Этот вопрос подробно затронут в [4].
	\item После устранения левой рекурсии можно применить левую факторизацию.
\end{enumerate}

\textbf{Вариант 3. Преобразование в грамматику без $\varepsilon$-правил}

\textbf{Определение.}
Назовем КС-грамматику $G = (N, \Sigma, P, S)$ грамматикой без $\varepsilon$-правил (или неукорачивающей), если либо

\begin{enumerate}
	\item $P$ не содержит $\varepsilon$-правил, либо
	\item есть точно одно $\varepsilon$-правило $S \to \varepsilon$ и $S$ не встречается в правых частях остальных правил из $P$.
\end{enumerate}

Постройте программу, которая в качестве входа принимает произвольную КС-грамматику $G = (N, \Sigma, P, S)$ и преобразует её в эквивалентную КС-грамматику $G' = (N', \Sigma', P', S')$ без $\varepsilon$-правил.

\textit{Указания.}
Воспользоваться алгоритмом 2.10. [1].
При тестировании воспользоваться примером 2.23. и упражнением 2.4.11. [1].

\section*{Текст программы}

С полным текстом программы можно ознакомиться по адресу: \url{https://github.com/Volkovaan/CD/tree/lab02/lab02/src}.

\section*{Тестирование и результаты}

Для тестирования использовались грамматики из примеров 4.7 и 4.9 [2], 2.23 [1], упражнений 2.29 [2], 2.4.11[1], а также несколько грамматик, найденных в примерах из сети интернет.

$G_1$ — исходная грамматика с правилами вывода $P_1$:
\begin{equation}
	\begin{aligned}
		S &\to Aa | b \\
		A &\to Ac | Aad | bd | \varepsilon \\
	\end{aligned}
\end{equation}

$G_1'$ — грамматика $G_1$ с устранённой левой рекурсией, $P_1'$:
\begin{equation}
	\begin{aligned}
		S &\to Aa | b \\
		A &\to bdA' | A' \\
		A' &\to cA' | adA' | \varepsilon \\
	\end{aligned}
\end{equation}

$G_2$ — исходная грамматика с правилами вывода $P_2$:
\begin{equation}
	\begin{aligned}
		E &\to E + T | T \\
		T &\to T * F | F \\
		F &\to (E) | a \\
	\end{aligned}
\end{equation}

$G_2'$ — грамматика $G_2$ с устранённой левой рекурсией, $P_2'$:
\begin{equation}
	\begin{aligned}
		E &\to TE' \\
		E' &\to +TE' | \varepsilon \\
		T &\to FT' \\
		T' &\to *FT' | \varepsilon \\
		F &\to (E) | a \\
	\end{aligned}
\end{equation}

$G_3$ — исходная грамматика с правилами вывода $P_3$:
\begin{equation}
	\begin{aligned}
		S &\to ABCd \\
		A &\to a|\varepsilon \\
		B &\to AC \\
		C &\to c|\varepsilon \\
	\end{aligned}
\end{equation}

$G_3'$ — грамматика $G_3$ без $\varepsilon$-правил, $P_3'$:
\begin{equation}
	\begin{aligned}
		S &\to Ad|ABd|ACd|ABCd|Bd|BCd|Cd|d \\
		A &\to a \\
		B &\to A|AC|C \\
		C &\to c \\
	\end{aligned}
\end{equation}

$G_4$ — исходная грамматика с правилами вывода $P_4$:
\begin{equation}
	\begin{aligned}
		S &\to aSbS|bSaS|\varepsilon \\
	\end{aligned}
\end{equation}

$G_4'$ — грамматика $G_4$ без $\varepsilon$-правил, $P_4'$:
\begin{equation}
	\begin{aligned}
		S' &\to S|\varepsilon \\
		S &\to aSbS|bSaS|aSb|abS|ab|bSa|baS|ba \\
	\end{aligned}
\end{equation}

$G_5$ — исходная грамматика с правилами вывода $P_5$:
\begin{equation}
	\begin{aligned}
		S &\to ABC \\
		A &\to BB|\varepsilon \\
		B &\to CC|a \\
		C &\to AA|b \\
	\end{aligned}
\end{equation}

$G_5'$ — грамматика $G_5$ без $\varepsilon$-правил, $P_5'$:
\begin{equation}
	\begin{aligned}
		S' &\to S|\varepsilon \\
		S &\to A|AB|ABC|B|BC|C \\
		A &\to B|BB \\
		B &\to C|CC \\
		C &\to A|AA \\
	\end{aligned}
\end{equation}


\section*{Выводы}

В ходе выполнения лабораторной работы были выполнены следующие задачи:

\begin{enumerate}
	\item Преисполнился:
	\begin{enumerate}
		\item соглашениями об обозначениях, принятые в литературе по теории формальных языков и грамматик и кратко описанные в приложении;
		\item основными понятиями и определениями теории формальных языков и грамматик;
		\item алгоритмом устранения левой рекурсии.
	\end{enumerate}
	\item Разработана, протестирована и отлажена программа устранения левой рекурсии.
	\item Разработана, протестировать и отлажена программа преобразования грамматики в соответствии с предложенным вариантом.
\end{enumerate}

\section*{Список использованной литературы}

\begin{enumerate}
	\item Ахо А., Ульман Дж. Теория синтаксического анализа, перевода и компиляции: В 2-х томах. Т. 1.: Синтаксичечкий анализ. - М.: Мир, 1978.
	\item Ахо А. В, Лам М. С., Сети Р., Ульман Дж. Д. Компиляторы: принципы, технологии и инструменты. – М.: Вильямс, 2008.
	\item Бунина Е. И., Голубков А. Ю. Формальные языки и грамматики. Учебное пособие. – М.: Изд-во МГТУ им. Н. Э.Баумана, Москва, 2006.
\end{enumerate}

\end{document}
